\section{Реализация}

\paragraph*{} Проектът е реализиран като приложение за командния ред. Поддържа се следния интерфейс

\verb|./parallel_dfs gen -n 20 -m 40 -t 4 --algo par_mat|

\begin{itemize}
\item \verb|gen| - режим на работа. Единственият имплементиран е gen - генерирай случаен граф и направи обхождане по него.
\item \verb|-n 20| - брой върхове
\item \verb|-m 40| - брой ребра
\item \verb|-t 4| - колко нишки да използва
\item \verb|--algo par_mat| - кои алгоритми и стуктури от данни да използва. Вариантите са \verb|seq_list|, \verb|par_list|, \verb|cheat_list|, \verb|seq_mat|, \verb|par_mat|, \verb|cheat_mat|
\end{itemize}

Всички аргументи могат да се видят с \verb|./parallel_dfs --help|

\paragraph*{} Следните изисквания към проекта не са имплементирани:
\begin{itemize}
\item въвеждане на граф от входен файл
\item извеждане на резултата в изходен файл (има опция за извеждане на резултата на стандартния изход в "debug" формат).
\end{itemize}

\paragraph*{} Използван е езикът Rust. Използвани са следните външни библиотеки:
\begin{itemize}
\item \verb|crossbeam| - lock free структури от данни
\item \verb|spin| - имплементация на Spin Lock
\item \verb|structopt| - обработка на аргументи от командния ред
\item \verb|rand| - генератори на случайни числа
\item \verb|rayon| - библиотека за паралелна обработка на данни
\end{itemize}
